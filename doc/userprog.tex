\documentclass[a4paper,12pt]{article}
\usepackage [utf8]{inputenc}
\usepackage{xspace} %better spacing option with :
\usepackage{amsmath,amsthm,amssymb} %package matematici
\usepackage{amsfonts} % altri simbolo matematici stronzi
\usepackage{latexsym} %altri simboli chissa quali
\usepackage{graphicx}
\usepackage{listings}
\usepackage{fancyhdr}
\usepackage{multicol}
\usepackage[pdftex]{color}
\usepackage{url}
\usepackage{subfigure}
\usepackage[final]{pdfpages}
\newcommand{\fncyblank }{\fancyhf {}}

\addtolength{\hoffset}{-1,5cm}
\addtolength{\textwidth}{3cm}

\definecolor{Gray}{cmyk}{0,0,0,0.50}

\usepackage{sectsty}
\sectionfont{\large}
\subsectionfont{\normalsize}

\title{Pintos Task 1: User Programs Design Document}
\author{Group 07: Ben Sheng Tan, Adanna Akwataghibe, Mark Aduol, Alessio Incitti }

\begin{document}
\maketitle

\section{ARGUMENT PASSING}

\subsection{ DATA STRUCTURES}

\subsection*{A1: (2 marks) }

\textit{Copy here the declaration of each new or changed ‘struct’ or ‘struct’ member, global or static variable, ‘typedef’, or enumeration. Identify the purpose of each in 25 words or less.}
\\ \\
In thread.h, we added:
\begin{lstlisting}
    struct thread
      {
        /* TASK 1 : Priority Checking */
        int base_priority;                /* initial priority of thread */
        struct list_elem donation_thread; /* thread element that is donated */
        struct lock* lock_waiting;        /* lock that thread is waiting for */
        struct list threads_donated;      /* list of threads */

        ...
      };

\end{lstlisting}

Purpose of each new addition:
\\ \\
\texttt{base\_priority}: The base priority of the thread, without taking priority donations into account.


\subsection{ALGORITHMS}

\subsection*{A2: (8 marks) }

\textit{Briefly describe how you implemented argument parsing. In particular you
should mention how you arrange for the elements of argv[] to be in the right
order and how you avoid overflowing the stack page?}

\\ \\TO WRITE

\subsection{RATIONALE}

\subsubsection*{A3: (5 marks) }
\textit{Why does Pintos implement \texttt{strtok\_r()} but not strtok()?}
\\ \\TOWRITE


\subsubsection*{A4: (5 marks) }
\textit{In Pintos, the kernel separates commands into a executable name and
arguments. In Unix-like systems, the shell does this separation. Identify at
least two advantages of the Unix approach.}
\\ \\TOWRITE

\section{SYSTEM CALLS}

\subsection{DATA STRUCTURES}

\subsection*{B1: (10 marks) }

\textit{Copy here the declaration of each new or changed ‘struct’ or ‘struct’
member, global or static variable, ‘typedef’, or enumeration. Identify the
purpose of each in 25 words or less.}
\\\\
In thread.h, we added:
\begin{lstlisting}
    struct thread
      {
        ...
        /* TASK 1: Advanced scheduling */
        int cpu_num;
        int nice;
        ...
        };
\end{lstlisting}

In 'thread.c', we added:
\begin{lstlisting}
    static int load_avg;
\end{lstlisting}

Purpose of each new addition:

\texttt{int cpu\_num}: the recent CPU value of the current thread in fixed point.

\texttt{int nice}: niceness value of the current thread.

\texttt{int load\_avg}: the system's load average value in fixed point.

\subsubsection*{B2: (5 marks) }
\textit{Describe how you associate file descriptors with open files. Are your
file descriptors unique within the entire OS or just within a single process? }
\\ \\ TOWRITE

\subsection{ALGORITHMS}

\subsubsection*{B3: (5 marks) }
\textit{Describe how your code ensures safe memory access of user provided data
from within the kernel. }
\\ \\TOWRITE

\subsubsection*{B4: (5 marks) }
\textit{Suppose a system call passes the kernel a pointer to a full page (4,096
bytes) of data that has to be copied from user space. What is the least and the
greatest possible number of inspections of the page table (e.g. calls to
\texttt{pagedir\_get\_page())} that might result? What about for a system
 call that passes a pointer to only 2 bytes of data? How might you improve these
numbers? }
\\ \\TOWRITE

\subsubsection*{B5: (8 marks) }
\textit{Describe your implementation of the "wait" system call and how it interacts
with process termination for both the parent and child.}
\\ \\ TOWRITE

\subsubsection*{B6: (5 marks) }
\textit{Any access to user program memory at a user-specified address can fail due to
a bad pointer value. Such accesses must cause the process to be terminated.
System calls are fraught with such accesses, e.g. a "write" system call
requires reading the system call number from the user stack, then each of the
call's three arguments, then an arbitrary amount of user memory, and any of
these can fail at any point. This poses a design and error-handling problem:
how do you best avoid obscuring the primary function of code in a morass of
error-handling? Furthermore, when an error is detected, how do you ensure
that all temporarily allocated resources (locks, buffers, etc.) are freed?
In a paragraph, describe the strategy or strategies you adopted for managing
these issues.}
\\ \\TOWRITE

\subsection{SYNCHRONIZATION}

\subsubsection*{B7: (5 marks) }
\textit{The "exec" system call returns -1 if loading the new executable fails, so it
cannot return before the new executable has completed loading. How does your
code ensure this? How is the load success/failure status passed back to the
thread that calls "exec"?}
\\ \\TOWRITE

\subsubsection*{B8: (5 marks) }
\textit{Consider parent process P with child process C. How do you ensure proper
synchronization and avoid race conditions when P calls wait(C) before or
after C exits? How do you ensure that all resources are freed in each case?
How about when P terminates, without waiting, before or after C exits? Your
answer must cover all four cases mentioned above and address resource freeing.}
\\ \\TOWRITE

\subsection{RATIONALE}

\subsubsection*{B9: (5 marks) }
\textit{Why did you choose to implement safe access of user memory from the
kernel in the way that you did?}
\\ \\ TOWRITE

\subsubsection*{B10: (5 marks) }
\textit{What advantages or disadvantages can you see to your design for file
descriptors?}
\\ \\ TOWRITE

\subsubsection*{B11: (2 marks) }
\textit{The default \texttt{tid\_t} to \texttt{pid\_t} mapping is the identity mapping. Did you
change this? Why?}
\\ \\ TOWRITE



\end{document}

\documentclass{article}

\usepackage{fullpage}
\usepackage{listings}
\usepackage{color}

\setlength{\parskip}{0.5em}

\definecolor{dkgreen}{rgb}{0,0.6,0}
\definecolor{gray}{rgb}{0.5,0.5,0.5}
\definecolor{mauve}{rgb}{0.58,0,0.82}

\lstset{
  language=C,
  aboveskip=3mm,
  belowskip=3mm,
  showstringspaces=false,
  columns=flexible,
  basicstyle={\small\ttfamily},
  numbers=none,
  numberstyle=\tiny\color{gray},
  breaklines=true,
  breakatwhitespace=true,
  tabsize=3
}


\title{Pintos Task 3: Virtual Memory Design Document}
\author{Group 07: Ben Sheng Tan, Adanna Akwataghibe, Mark Aduol, Alessio Incitti }

\begin{document}
\maketitle

\section{PAGE TABLE MANAGEMENT}

\subsection{ DATA STRUCTURES}

\subsection*{A1: (5 marks) }

\textit{Copy here the declaration of each new or changed ‘struct’ or ‘struct’ member, global or static variable, ‘typedef’, or enumeration. Identify the purpose of each in 25 words or less.}
\\ \\
In process.c, we added:
\begin{lstlisting}
	#define MAX_ARGS 50		   /* Maximum number of arguments for argument parsing */
\end{lstlisting}
No additional structs were added or changed for argument parsing.

\subsection{ALGORITHMS}

\subsection*{A2: (5 marks) }

\textit{Describe your code for locating the frame, if any, that contains the data of a given page.}
\\ \\
TOWRITE


\subsubsection*{A3: (5 marks) }
\textit{How does your code coordinate accessed and dirty bits between
 kernel and user virtual addresses that alias a single frame?
 Your answer should include a discussion of how you have implemented
 sharing of read only pages.}
\\ \\
TOWRITE

\subsection{SYNCHRONIZATION}

\subsubsection*{A4: (5 marks) }
\textit{When two user processes both need a new frame at the same time,
how are races avoided? You should consider both when there are
and are not free frames available in memory.}
\\ \\
TOWRITE

\subsection{RATIONALE}

\subsubsection*{A5: (5 marks) }
\textit{Why did you choose the data structure(s) that you did for representing virtual-to-physical mappings?.}
\\ \\
TOWRITE


\section{PAGING TO AND FROM DISK}

\subsection{DATA STRUCTURES}

\subsection*{B1: (5 marks) }

\textit{Copy here the declaration of each new or changed ‘struct’ or ‘struct’
member, global or static variable, ‘typedef’, or enumeration. Identify the
purpose of each in 25 words or less.}
\\\\
In thread.h, we added:
\begin{lstlisting}
    struct thread
                                 each thread */
          };
\end{lstlisting}

\subsection{ALGORITHMS}

\subsubsection*{B2: (5 marks) }
\textit{When a frame is required but none is free, some frame must be evicted.  Describe your code for choosing a frame to evict.}
\\ \\
TOWRITE


\subsubsection*{B3: (5 marks) }
\textit{When a process P obtains a frame that was previously used by a
process Q, how do you adjust the page table (and any other data
structures) to reflect the frame Q no longer has? }
\\ \\
TOWRITE


\subsubsection*{B4: (5 marks) }
\textit{Explain your heuristic for deciding whether a page fault for an
invalid virtual address should cause the stack to be extended into
the page that faulted. }
\\\ \\
TOWRITE



\subsection{SYNCHRONIZATION}

\subsubsection*{B5: (8 marks) }
\textit{Explain the basics of your VM synchronization design.  In particular, explain how it prevents deadlock.  (Refer to a
 textbook for an explanation of the necessary conditions for deadlock.)}
\\ \\
TO WRITE


\subsubsection*{B6: (5 marks) }
\textit{A page fault in process P can cause another process Q's frame
to be evicted.  How do you ensure that Q cannot access or modify
the page during the eviction process?  How do you avoid a race
between P evicting Q's frame and Q faulting the page back in?
}
\\ \\
TOWRITE


\subsubsection*{B7: (5 marks) }
\textit{Explain how you handle access to paged-out pages that occur
during system calls. How do you gracefully handle attempted
accesses to invalid virtual addresses?}
\\ \\
TOWRITE

\subsection{RATIONALE}

\subsubsection*{B8: (5 marks) }
\textit{A single lock for the whole VM system would make
synchronization easy, but limit parallelism.  On the other hand,
using many locks complicates synchronization and raises the
possibility for deadlock but allows for high parallelism.  Explain
where your design falls along this continuum and why you chose to
design it this way.}
\\ \\
TOWRITE


\section{MEMORY MAPPED FILES}

\subsection{DATA STRUCTURES}

\subsection*{C1: (5 marks) }

\textit{Copy here the declaration of each new or changed ‘struct’ or ‘struct’
member, global or static variable, ‘typedef’, or enumeration. Identify the
purpose of each in 25 words or less.}
\\\\
In thread.h, we added:
\begin{lstlisting}
    struct thread
                                 each thread */
          };
\end{lstlisting}

\subsection{ALGORITHMS}

\subsubsection*{C2: (5 marks) }
\textit{Describe how memory mapped files integrate into your virtual
 memory subsystem.  Explain how the page fault and eviction
 processes differ between swap pages and other pages.}
\\ \\
TOWRITE


\subsubsection*{C3: (5 marks) }
\textit{Explain how you determine whether a new file mapping overlaps any existing segment. }
\\ \\
TOWRITE


\subsection{RATIONALE}

\subsubsection*{C4: (5 marks) }
\textit{Mappings created with "mmap" have similar semantics to those of
data demand-paged from executables, except that "mmap" mappings are
written back to their original files, not to swap.  This implies
that much of their implementation can be shared.  Explain why your
implementation either does or does not share much of the code for
the two situations.}
\\\ \\
TOWRITE



\end{document}

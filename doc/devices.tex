\documentclass{article}

\usepackage{fullpage}
\usepackage{listings}
\usepackage{color}

\setlength{\parskip}{0.5em}

\definecolor{dkgreen}{rgb}{0,0.6,0}
\definecolor{gray}{rgb}{0.5,0.5,0.5}
\definecolor{mauve}{rgb}{0.58,0,0.82}

\lstset{
  language=C,
  aboveskip=3mm,
  belowskip=3mm,
  showstringspaces=false,
  columns=flexible,
  basicstyle={\small\ttfamily},
  numbers=none,
  numberstyle=\tiny\color{gray},
  breaklines=true,
  breakatwhitespace=true,
  tabsize=3
}

\title{Pintos Task 0: Design Document}
\author{Ben Sheng Tan}

\begin{document}
\maketitle

\subsection*{---- DATA STRUCTURES ----}

\subsection*{A1: (2 marks) }

\textit{Copy here the declaration of each new or changed ‘struct’ or ‘struct’ member, global or static variable, ‘typedef’, or enumeration. Identify the purpose of each in 25 words or less.}
\\ \\
A \texttt{"sleep\_sema"} in \texttt{devices/timer.c} is semaphores used to block the sleeping threads. Ordered by increasing number of \texttt{wake\_up} tick semaphores and initialised in \texttt{timer\_init()}.

\begin{lstlisting}
    /* TASK O: Semaphores used to block the sleeping threads */
    static struct list sleep_sema;
\end{lstlisting}
\hfill \break
Added to struct thread:

\begin{lstlisting}
    int64_t wake_up_tick;  /* Keep track the tick when sleeping thread wakes up */
\end{lstlisting}


\subsection*{---- ALGORITHMS ----}

\subsubsection*{A2: (2 marks) }
\textit{Briefly describe what happens in a call to timer\_ sleep(), including the actions performed by the timer interrupt handler on each timer tick.}
\\ \\
When \texttt{timer\_sleep()} is called, ticks is incremented by the current number of timer ticks and the value is used to update the \texttt{wake\_up\_tick} of current thread. A new semaphore is created to make the current thread sleeping. Interrupts are disabled before inserting the semaphore into the list of sleep\_sema, which is in ascending order of the number of ticks the thread sleeps for. This is to ensure that other threads cannot insert into the list at the same time and disrupt the list ordering. The list ordering is also very important in regards to the interrupt handler; \texttt{timer\_interrupt ()}, to check
whether the sleeping thread needs to be woken up.
\par
When \texttt{timer\_interrupt()} is called, tick is incremented and the function iterates through the sleep\_sema. It retrieves the woken thread first, as sleep\_sema is  in ordered, and return if we haven't passed the \texttt{wake\_up} tick yet. Later perform a sema\_up, which will unblock the sleeping thread before removing the semaphore from the list, and processing the next semaphore.

\subsubsection*{A3: (2 marks) }
\textit{What steps are taken to minimize the amount of time spent in the timer interrupt handler?}
\\ \\
Having the list of semaphores ordering with the earliest wake up tick first minimises the time spent in the interrupt handler routine as there is no need to check those that don’t need to be woken up yet. Whenever a thread has a wakeup tick bigger than the current system tick, we can stop searching for the next thread that wakes up and terminate the loop.
\par
The ordering is defined by the \texttt{less\_sema\_ticks} function in  \texttt{devices/timer.c}. This
compares the \texttt{wake\_up\_tick}s of two \texttt{thread} \texttt{list\_elem}s and return \texttt{true} if the first one is smaller than the second one. In \texttt{threads/thread.c},  \texttt{thread\_for\_sema\_list\_elem} function is implemented to return the thread associated with a semaphore element of a list.


\subsection*{---- SYNCHRONIZATION ----}

\subsubsection*{A4: (1 mark) }
\textit{How are race conditions avoided when multiple threads call timer\_sleep() simultaneously?}
\\ \\
The \texttt{list\_elem} and \texttt{wake\_up\_tick} in the thread structs are independent of each other. The only shared data structure is the \texttt{sleep\_sema}. By disabling interrupts before adding to the list and enabling them after, mutations to this are atomic to prevent any time slice happening.

\subsubsection*{A5: (1 mark) }
\textit{How are race conditions avoided when a timer interrupt occurs during a call to timer\_sleep()?}
\\ \\
Race conditions are avoided by disabling interrupts when adding the semaphores to \texttt{sleep\_sema} during a call to timer\_sleep(). Since it is changing the global state and \texttt{timer\_interupt},  mutation to the list need to be  atomically actions. For this case, timer interrupts cannot acquire a lock on the list.

\subsection*{---- RATIONALE ----}

\subsubsection*{A6: (2 marks) }
\textit{Why did you choose this design? In what ways is it superior to another design you considered?}
\\ \\
Since synchronisation primitives were already provided, it is safer to use semaphores to handle the thread blocking/unblocking, rather than those low level functions, \texttt{thread\_block} and \texttt{thread\_unblock}. To apply this method, struct \texttt{semaphore\_elem} is declared globally in \texttt{src/threads/sych.h}.
\par
Also, I optimised my approach by storing the list of semaphores in the ordering with the earliest wake up tick first to minimize time spent in interrupt handlers. This prevent me from needlessly iterating over the list and finding threads which don’t have to be unblocked. This is vital as in Pinto, \texttt{timer\_interrupt} is called frequently.
\end{document}
